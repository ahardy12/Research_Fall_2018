\vspace{2in}
\begin{abstract}

Predicting trends and movements within a given industry sector can be an incredibly difficult, time consuming and potentially fruitless endeavor. The goal of the research conducted over this semester is to show the potential to use data analysis and eventually machine learning techniques to protect changes within a given industry. Python was used to pull information from the Wharton database the most connected companies were determined using graphing methods. Once the companies were selected a set of variables were compiled to represent how well a company was doing. The variables were standardized and PCA reduction was used to reduce the variables to two principle components. Once the graphs were completed, companies were grouped by sector to establish if any similarities existed between sectors. Several sectors had graphes with similar changes in trajectory. Upon looking at the events surronding changes it trajectory it was discovered that each change in trajectory on the graph corresponded to a major event affecting the company. Each of the companies narrtives have been examined to be the basis for moving from a descriptive model to the ultimate goal of predictive model of industry forces.
\end{abstract} 
