\chapter{Indstrusy Research}

\section{Introduction}

	The research began by looking at the supply chain database created by Wharton. This database contained a list of companies and who they supplied to organized by year using the unique identifier gvkey. Using the gvkeys of the supplier as the source and the gvkeys of customers as destinations a directed graph was created to determine the most connected companies for each year. Throughout the years several companies appeared over and over. 
\section{Process}

	This small group of regularly highly connected companies became the base group for our research. Twenty-seven companies were divided into nine sectors: Automotive, Retail, Oil, Computers, Hardware, Telecommunication and Conglomerates. Several variables were selected to represent how well the companies were doing. Some of the variables included were total assets, net income, total sales and stock holder equity.\\
\newline

	Once the variables were compiled, they were standardized and a PCA reduction was performed. This allowed all twenty-seven companies to be compared side by side. Trends were noticed immediately regarding General Motors and Ford Motor Company. Once the selected companies were divided into smaller graphs, still using the original standardization and PCA from the large graph, more trends were noticed. \\
\newline

	One of the most striking trends within in an industry was the similarities between Home Depot and Lowe’s. The graphs had similar trajectories and then both changed trajectories in the same direction around the same time and then once again changed back to the original direction at the same time. Both changes in direction coincide with the begin and end of the house market crash. The oil industry has a whole has a lot of similar trajectories in the graphs for each of the companies, most noticeably between Exxon and Shell. \\
\newline

	As research began into the history of each of the companies and the events transpiring during each of the changes in trajectory one major theme occurred over and over. Nearly every change in a company’s trajectory coincided with a major event for the company. From the housing market crash in the hardware industry to AT\&T being bought by what was at the time a spin off company to the oil industry reacting to the changes in oil prices in the mid 2000’s.\\
\newline

	The discovery of the changes in graph trajectories has led to an increased emphasis on telling the story of each company.  The following sections of this paper are narratives of the major changes in each company’s graph and how those changes are representative of major events affect each company. \\